\section{Machine Learning and Deep Learning}
For the past decades, Machine Learning has become one of the most powerful tools, allowing individuals to perform complex analyses for better insights. The technique is widely applied and brings significant benefits in various vital fields, such as business, healthcare, agriculture or traffic management, etc.
Differ from the traditional programming paradigm, where the engineers have to craft the input data themselves and propose many rules to produce the most satisfying answers. \\
However, when the data increases extremely large, it contains many latent trends and patterns, which usually diminish hand-crafted solutions to get potential results.
This is where the Machine Learning techniques take place. The main goal of the approach is to develop a system that receives our observed data and corresponding expected answers, then return a set of rules that best fit those samples. These rules can be utilized to make predictions for new input. 
The machine-learning system applies some transformations on the input data to extract frequential or special patterns, which are usually called features, then use them to predict the output. The approach’s performance is then evaluated by appropriate metrics for the given task. Based on these scores, the system can choose the best transformation for each problem among a set of predefined functions (also known as hypothesis space). \\
As a subfield of machine learning, deep learning is formed as complex neural network architectures, containing more processing layers to learn hidden representations of the input data. 
This improvement is constructive in perceptual problems, where hand-crafted filters are merely impossible to achieve good results. 
In the following sections, we first discuss the beginning but original neural network to the further advanced ones.
