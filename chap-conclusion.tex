\chapter{Conclusion}
\label{chap-conclusion}
\begin{ChapAbstract}
In this chapter, we re-declare the problem of abnormalities finding and anatomical landmarks detection together with nuclear segmentation using deep convolution neural networks. Overview of what researches we conducted as well as analysis of future works are also discussed to provide the overall summary of our thesis. 
\end{ChapAbstract}

\section{Results}
During the progress of doing thesis's researches, we gain a huge knowledge of deep neural networks, different techniques used in image classification, object detection, i.e Faster R-CNN, as well as recent state-of-the-art semantic and instance segmentation networks. Additionally, we also experienced how to deploy deep learning models on Linux operating system, using different cloud computing services i.e Google Cloud, Google Colab, etc. while conducting experiments, i.e Faster R-CNN, Residual Neural Network, etc.

Medico image classification is a challenging problem because of the fine-grained images, less training data and require high accuracy. In our current approach, we focus on training a combination of Residual Neural Network and Faster R-CNN with different modifications of the training set. Additionally, object detection method is applied to detect small symptoms of diseases, which are useful evidences for the classification task. Accuracy and inference time that we reach is acceptable and appropriate for real-time constraint. All the experimented methods and our main improvements, are evaluated during the Medico: The 2018 Multimedia for Medicine Task as well as the The Biomedia ACM MM Grand Challenge 2019. As can be seen in table \ref{teams_result}, our proposed method achieves the highest $MCC$ score, which is 0.9424 while the accuracy at 99.33\%.

In the segmentation work, we have shown the value of several enhancements to the standard U-Net architecture, namely, encoding rotation and translation equivariance and adding additional residual blocks, and our novel data augmentation
method for automated nuclear segmentation in histology
images. On the entire test set, our method
achieved an AJI of 0.629. This work can be considered as a parallel work to the many other current developments in deep-learning based nuclear segmentation, which also could be incorporated to further improve performance. By contributing to
improved performance for this crucial step for many computational pathology pipelines, without adding significant
computational overhead, we believe these enhancements
will help to enable future discoveries in pathology, with
the hope of increasing the clinical impact of computational
pathology.

\section{Future Works}
Regarding to the future works in order to improve our current diagnosis system, we need a more robust approach to exploit the distinction between easy-confused classes, e.g, \textit{esophgitis} and \textit{normal-z-line}, or \textit{dyed-lifted-polyps} and \textit{dyed-resection-margins}. As mentioned before, this is a challenging problem since the similarity between these classes makes it infeasible to be distinguished just by vision-based only. Further research need to be conducted in order to utilize different source of medical tests to make the final decision of our system.

Besides, integrating attention mechanisms can help this system to focus more on abnormalities signal and not only make the prediction explainable but also assist endoscopists to localize and pay more attention to some specific regions of an endoscopic image. Other approaches like low-level and handcrafted features can also be applied together in the detection process, while it is more explainable and and reliable than deep-learning approach in some special cases. We can further develop our thesis by executing on more datasets.

Last but not least, abnormalities finding and anatomical landmarks detection is only an initialization step of a complete multimedia assisted diagnosis systems. More experiments need to be done, especially on different source of endoscopic images to evaluate other factors such as illumination conditions, devices calibration, image quality that could effect the performance of our system. Nevertheless, we also need to integrate our  detection module to some visualization system that allows endoscopists to interact and improve the endoscopy examination process. Surveys and interviews conducted on special  faculty member from medical school and hospital is necessary to understand and meet the needs and requirements of real users.

In segmentation work, the formulation of joint segmentation and cancer diagnosis/grade prediction would be interesting and have huge potential to the real-life application. Segmentation provides extraction of high-quality features for nuclear morphometrics, appearance features and other analysis in computational pathology such as density, nucleus-to-cytoplasm ratio, average size, pleomorphism, etc. Building a CADx system aggregating these data to provide useful information that can support pathologist or doctor to assess not only cancer grades but also for predicting treatment effectiveness will bring a lot of benefit to the health care community. Besides, we also would like to extend our enhancements to other related fields such as astronomy image analysis, aerial photography analysis, etc, which shares some similarities to our segmentation work on H\&E stained histopathology images. 


\begin{ChapAbstract}
%In this chapter, we briefly discuss what we gain during doing our thesis, from theoretical perspective to technological aspects. The overall achievements, in both abnormal findings \& anatomical landmarks detection together and nuclear segmentation, are described. Some directions are also illustrated to show some potentialities of extending our thesis in the future works.  

\end{ChapAbstract}