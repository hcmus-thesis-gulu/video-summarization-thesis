\subsection{Problem Statement}
\label{subsec:rel-statement}

Video summarization aims to generate a concise overview of video content by selecting the most informative and significant parts. The resulting summary can take the form of either a set of representative video frames, known as a video storyboard, or a compilation of video fragments stitched together in chronological order, referred to as a video skim. Video skims have an advantage over static frame sets as they can include audio and motion elements, allowing for a more natural storytelling experience and potentially conveying more information. Moreover, watching a video skim is often more engaging and captivating for viewers compared to a slideshow of frames \cite{Li2001Overview}. On the other hand, storyboards offer greater flexibility in terms of data organization for browsing and navigation purposes, as they are not bound by timing or synchronization constraints \cite{Calic2007Comic,Wang2007VideoCollage}.

Our problem statement aligns closely with the concept of video storyboards, which involve selecting a subset of representative video frames to summarize the content. By focusing on these key frames, we aim to capture the essence and important aspects of the video in a condensed form. This approach allows for efficient browsing and navigation through the video data while providing a comprehensive overview of its content.