\section{Project content}
% \section{Thesis content}
\label{sec:thesis_content}

After \textbf{Chapter 1}, the remainder of our project is composed of four chapters as follows:

% After \textbf{Chapter 1}, the remainder of our thesis is composed of four chapters as follows:

\textbf{Chapter 2: Related Works}

In this chapter, we first introduce an overview of the main approaches for the Interactive Retrieval System and previous versions of our system.
We then discuss several methods used in Referring Expression Segmentation tasks. Finally, we discuss state-of-the-art approaches in Semi-supervised Video Object Segmentation that we leverage as a post-process to further boost our performance.

\textbf{Chapter 3: Flexible Interactive Retrieval System}

In this chapter, we go into detail about our retrieval system called \textbf{FIRST}. First, we outline the overview of our retrieval system. Following sections describe the design of our system and AI-related components in retrieval process. Finally, we provide more details about the Lifelog Search Challenge 2022 and Visual Browser Showdown 2022, the evaluation metrics used in these challenges, and the result of our interactive retrieval system on each challenge.

\textbf{Chapter 4: Referring Expression Segmentation}

In this chapter, we present our Referring Expression Segmentation framework \textbf{VLFormer} in detail. We first introduce an overview of our framework. Next, we describe each part of our architecture and how efficiently we train the network. Then, an extension to video data is provided. Finally, we discuss about challenges we participated in, the datasets we conducted the experiments, and analyze our performance in both quantitative and qualitative aspects. 

\textbf{Chapter 5: Conclusion}

In this chapter, we summarize our work and briefly discuss the potential improvements of the current approach for future research.