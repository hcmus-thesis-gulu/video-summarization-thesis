\chapter{Related Work}
\label{chapter:related}

\begin{ChapAbstract}
  Deep learning methods have dominated the video summarization task for a long time due to their remarkable ability to automatically learn relevant features and representations from large-scale video data. In this chapter, we initially cover the fundamental aspects, encompassing the problem statement, datasets used, and evaluation metrics employed in video summarization research in Section \ref{section:rel-preliminary}. Subsequently, we conduct a thorough examination of the existing literature in video summarization, emphasizing three principal categories of approaches: supervised methods (Section \ref{section:rel-supervised}), unsupervised methods (Section \ref{section:rel-weakly}), and weakly supervised methods (Section \ref{section:rel-unsupervised}).
\end{ChapAbstract}

\section{Preliminary}
\label{section:rel-preliminary}

\subsection{Problem Statement}
\label{subsec:rel-statement}

Video summarization aims to generate a concise overview of video content by selecting the most informative and significant parts. The resulting summary can take the form of either a set of representative video frames, known as a video storyboard, or a compilation of video fragments stitched together in chronological order, referred to as a video skim. Video skims have an advantage over static frame sets as they can include audio and motion elements, allowing for a more natural storytelling experience and potentially conveying more information. Moreover, watching a video skim is often more engaging and captivating for viewers compared to a slideshow of frames \cite{Li2001Overview}. On the other hand, storyboards offer greater flexibility in terms of data organization for browsing and navigation purposes, as they are not bound by timing or synchronization constraints \cite{Calic2007Comic,Wang2007VideoCollage}.

Our problem statement aligns closely with the concept of video storyboards, which involve selecting a subset of representative video frames to summarize the content. By focusing on these key frames, we aim to capture the essence and important aspects of the video in a condensed form. This approach allows for efficient browsing and navigation through the video data while providing a comprehensive overview of its content.

\subsection{Problem Formulation}
\label{subsec:rel-formulation}
Given an input video $\textbf{I}=\{I^{(t)}\}_{t=1}^T$ where each frame $I^{(t)} \in \mathbb{R} ^{C \times H \times W}$, the goal of video summarization using the storyboard approach is to generate a concise summary $\textbf{S}$ that consists of a subset of representative frames. The summary $\textbf{S}$ is denoted as $\{I^{(t_i)}\}^k_{i=1}$, where $k$ is typically much smaller than $T$ and $t_1 < t_2 < \dots < t_k$.

\subsection{Datasets}
\label{subsec:rel-datasets}

As referenced in Section \ref{section:intro-motivation}, two datasets that prevail in the video summarization bibliography are SumMe \cite{Gygli2014SumMe} and TVSum \cite{Song2015TVSum}. 

SumMe dataset comprises 25 videos, ranging from 1 to 6 minutes in duration, encompassing diverse content and captured from both first-person and third-person perspectives. Each video has been annotated by 15 to 18 users, resulting in multiple fragment-level user summaries. These summaries typically span 5\% to 15\% of the original video duration. 

TVSum dataset comprises 50 videos, with durations ranging from 1 to 11 minutes. These videos cover content from 10 categories of the TRECVid MED dataset. Each video in TVSum has been annotated by 20 users, providing shot- and frame-level importance scores on a scale of 1 to 5.

In addition to SumMe and TVSum, two common datasets for evaluating video summaries are OVP \cite{De2011VSUMM} and YouTube \cite{De2011VSUMM}. Each dataset comprises 50 videos, with annotations consisting of sets of key-frames generated by 5 users. The video durations span from 1 to 4 minutes for OVP and 1 to 10 minutes for YouTube. These datasets encompass a wide variety of video content, including documentaries, educational videos, ephemeral videos, historical footage, and lectures in the case of OVP, and cartoons, news clips, sports highlights, commercials, TV shows, and home videos in the case of YouTube.

Considering the size of these datasets, it is evident that there is a scarcity of large-scale annotated datasets, which limits their utility in enhancing the training of sophisticated supervised deep learning architectures.

Some less commonly used datasets for video summarization include CoSum \cite{Chu2015CoSum}, MED-summaries \cite{Potapov2014MEDSummaries}, Video Titles in the Wild (VTW) \cite{Zeng2016TitleWild}, League of Legends (LoL) \cite{Fu2017VideoLoL}, and FVPSum \cite{Ho2018FVPSum}. 

CoSum focuses on video co-summarization. It consists of 50 videos obtained from Youtube using 10 query terms related to the content of SumMe dataset. Each video has an approximate duration of 4 minutes, from which sets of key-fragments are selected by 3 different annotators.

MED-Summaries consists of 160 videos from TRECVID 2011 MED dataset. The dataset is divided into a validation set with 60 videos from 15 event categories and a test set with 100 videos from 10 event categories. The majority of videos has durations range from 1 to 5 minutes, with each being annotated with a set of importance scores, averaged over 1 to 4 annotators.

The VTW dataset consists of 18100 open domain videos, out of which 2000 videos are annotated at the sub-shot level with highlight scores. These user-generated videos are untrimmed and typically contain a highlight event. On average, the videos in the dataset have a duration of 1.5 minutes.

The LoL dataset consists of 218 long videos, ranging from 30 to 50 minutes in duration. These videos showcase game matches from the North American League of Legends Championship Series (NALCS). The annotations for this dataset are derived from a YouTube channel that features community-generated highlights, with the highlight videos typically having a duration of 5 to 7 minutes. As a result, there is one set of key-fragments available for each video in the dataset.

The FPVSum dataset focuses on first-person video summarization and comprises 98 videos, totaling over 7 hours of content. These videos are sourced from 14 categories of GoPro viewer-friendly videos. For each category, approximately 35\% of the video sequences have been annotated with ground-truth scores by at least 10 users, while the remaining sequences are considered unlabeled examples. This dataset provides valuable resources for evaluating and developing first-person video summarization algorithms.

Apostolidis \etal~\cite{Apostolidis2021Video} have compiled a comprehensive summarization table, showcasing the main characteristics of the aforementioned datasets. For reference, \hyperref[table:dataset-characteristics]{Table \ref{table:dataset-characteristics}} presents an overview of the dataset attributes, such as video count, annotation types, video duration, and user involvement.
\begin{table}
  \caption{Datasets for video summarization and their characteristics.}
  \scriptsize
  \begin{tabular}{|M{0.09\textwidth}|M{0.07\textwidth}|M{0.09\textwidth}|M{0.4\textwidth}|M{0.16\textwidth}|M{0.11\textwidth}|}
    \hline
    \bfseries Dataset & \bfseries no. videos & \bfseries duration (min) & \bfseries content & \bfseries type of annotations & \bfseries annotators per video \\ 
    [0.5ex] 
    \hline\hline
    SumMe \cite{Gygli2014SumMe} & 25 & 1 - 6 & holidays, events, sports & multiple sets of key-fragments & 15 - 18 \\
    \hline
    TVSum \cite{Song2015TVSum} & 50 & 2 - 10 & news, how-to's, user-generated, documentaries (10 categories - 5 video each) & multiple fragment level scores & 20 \\
    \hline
    OVP \cite{De2011VSUMM} & 50 & 1 - 4 & documentary, educational, ephemeral, historical, lecture & multiple sets of key frames & 5 \\
    \hline
    YouTube \cite{De2011VSUMM} & 50 & 1 - 10 & cartoons, sports, tv-shows, commercial. home videos & multiple sets of key frames & 5 \\
    \hline
    CoSum \cite{Chu2015CoSum} & 51 & ~ 4 & holidays, events, sports (10 categories) & multiple sets of key fragments & 3 \\
    \hline
    MED \cite{Potapov2014MEDSummaries} & 160 & 1 - 5 & 15 categories of various genres & one set of importance score & 1 - 4 \\
    \hline
    VTW \cite{Zeng2016TitleWild} & 2000 & 1.5 (avg) & user-generated videos that contain a highlight event & sub-shot level highlight scores & - \\
    \hline
    LoL \cite{Fu2017VideoLoL} & 218 & 30 - 50 & matches from a League of Legends tournament & one set of key fragments & 1 \\
    \hline
    FPVSum \cite{Ho2018FVPSum} & 98 & 4.3 (avg) & first-person videos (14 categories) & multiple frame level scores & 10 \\
    \hline
  \end{tabular}
  \label{table:dataset-characteristics}
\end{table}

% In this study, we will thoroughly analyze and employ these datasets as benchmarks for evaluating the performance of video summarization algorithms.

\subsection{Evaluation Metrics}
\label{subsec:rel-evaluation}
\section{Supervised approaches}
\label{section:rel-supervised}

This is motivation.
\section{Unsupervised approaches} 
\label{section:rel-unsupervised} 

Unsupervised methods eliminate the need for ground-truth data, which typically requires time-consuming and labor-intensive manual annotation. Instead, unsupervised approaches leverage large collections of original videos for training. Through learning mechanisms designed for unsupervised settings, these methods extract meaningful information from the video data to generate summaries.

\subsection{Fooling Discriminator to Discriminate Original Video from Summary-Based Reconstruction}
\label{section:rel-unsup-discriminative}

% Due to the absence of ground-truth data, unsupervised video summarization techniques leverage the principle that a representative summary should enable viewers to comprehend the original video content. To achieve this, Generative Adversarial Networks (GANs) are employed to learn the creation of video summaries that facilitate accurate reconstruction of the original video. The training process involves a Summarizer, consisting of a Key-frame Selector and a Generator. The Key-frame Selector estimates frame importance and generates a summary, while the Generator reconstructs the video based on the generated summary. By inputting the video frames and predicting frame-level importance scores, the Summarizer reconstructs the original video. The reconstructed video, alongside the original one, is fed into a trainable Discriminator that evaluates their similarity.

% Similar to supervised GAN-based methods, the training of the entire summarization architecture follows an adversarial approach. In this case, the Summarizer's objective is to deceive the Discriminator by making it challenging to distinguish between the summary-based reconstructed video and the original video. Conversely, the Discriminator aims to improve its discrimination abilities. When the discrimination becomes indistinguishable (i.e., similar classification error for both videos), the Summarizer successfully constructs a highly representative video summary. Notably, Mahasseni et al. (2017) combined an LSTM-based key-frame selector, a Variational Auto-Encoder (VAE), and a trainable Discriminator, using adversarial learning to minimize the distance between the original video and the summary-based reconstructed version.

% Building upon this foundation, Apostolidis et al. (2019) proposed a stepwise, label-based approach for training the adversarial part of the network, leading to enhanced summarization performance. Yuan et al. (2019) introduced an approach aiming to maximize the mutual information between the summary and the video, utilizing a pair of trainable discriminators and a cycle-consistent adversarial learning objective. Their frame selector, a bidirectional LSTM, constructs a video summary by modeling temporal dependencies among frames. The summary is then evaluated by two GANs—an encoder-decoder GAN for forward reconstruction and a backward GAN for reverse reconstruction. The consistency between these reconstructions quantifies information preservation, guiding the frame selector to identify the most informative frames for the video summary.

% In a subsequent work, Apostolidis et al. (2020) integrated an Actor-Critic model into a GAN, formulating the selection of important video fragments as a sequence generation task. The Actor and Critic engage in a game that incrementally selects video key fragments, with rewards from the Discriminator influencing their choices. This training workflow enables the Actor and Critic to learn a value function (Critic) and a policy for key-fragment selection (Actor). Other approaches extended the core VAE-GAN architecture by incorporating tailored attention mechanisms.

% For instance, Jung et al. (2019) proposed a VAE-GAN architecture extended with a chunk and stride network (CSNet) and a tailored difference attention mechanism, capturing frame dependencies at various temporal granularities during keyframe selection. In a subsequent work, Jung et al. (2020) introduced a self-attention mechanism combined with relative position modeling, decomposing the frame sequence into non-overlapping groups to capture both local and global interdependencies. Apostolidis et al. (2020) presented a variation of their prior work, replacing the VAE with a deterministic Attention Auto-Encoder to improve attention-driven reconstruction and key-fragment selection. He et al. (2019) proposed a self-attention-based conditional GAN, utilizing a conditional feature selector and a multi-head self-attention mechanism to focus on important temporal regions and model long-range dependencies in the video sequence. Finally, Rochan et al. (2019) developed an approach for video summarization from unpaired data, employing an adversarial process with GANs and a Fully-Convolutional Sequence Network (FCSN) encoder-decoder. The model aimed to learn a mapping function from raw video to a human-like summary, aligning the summary distribution with human-created summaries while ensuring content diversity through an applied constraint on the learned mapping function.

\subsection{Focusing on Specific Desired Properties}
\label{subsec:rel-unsup-specific-properties}

% Addressing the challenges of unstable training and limited evaluation criteria in GAN-based methods, certain unsupervised approaches focus on specific properties of an optimal video summary. These approaches employ reinforcement learning principles in conjunction with hand-crafted reward functions that quantify desired characteristics in the generated summary. Illustrated in Fig. 6, the Summarizer takes the video frame sequence as input and generates a summary by predicting frame-level importance scores. The predicted summary is then evaluated by an Evaluator, which employs hand-crafted reward functions to measure the presence of specific desired characteristics. The computed scores are combined to form an overall reward value, guiding the training of the Summarizer.

% The initial work in this direction, proposed by Zhou et al. (2018), formulates video summarization as a sequential decision-making process. They train a Summarizer to produce diverse and representative video summaries using a diversity-representativeness reward. The diversity reward quantifies the dissimilarity among selected keyframes, while the representativeness reward measures the visual resemblance of the selected keyframes to the remaining frames of the video.

% Expanding on this method, Yaliniz et al. (2021) present another reinforcement-learning-based approach that incorporates the uniformity of the generated summary. They employ Independently Recurrent Neural Networks (IndRNNs) activated by a Leaky ReLU function to model temporal dependencies among frames. This addresses issues related to decaying, vanishing, and exploding gradients in LSTM models and facilitates better learning of long-term dependencies. In addition to rewards associated with representativeness and diversity, Yaliniz et al. introduce a uniformity reward to enhance the coherence of the summary and prevent redundant jumps between selected video fragments.

% Gonuguntla et al. (2019) propose a method utilizing Temporal Segment Networks, originally designed for action recognition in videos, to extract spatial and temporal information from video frames. They train the Summarizer using a reward function that evaluates the preservation of the video's main spatiotemporal patterns in the generated summary.

% Lastly, Zhao et al. (2020) present a mechanism that combines video summarization and reconstruction. Video reconstruction aims to estimate how well the summary allows viewers to infer the original video, similar to some GAN-based methods. Video summarization is learned based on feedback from the reconstructor and the output of trained models that assess the representativeness and diversity of the visual content in the generated summary.

\subsection{Building Object-Oriented Summaries through Key Object Motion}
\label{subsec:rel-unsup-object-oriented}

% Zhang et al. (2018) [108] devised a novel method that prioritizes the retention of fine-grained semantic and motion details within the video summary. Their approach involves an initial preprocessing step aimed at identifying significant objects and their key motions. Leveraging this information, the method represents the entire video by creating segmented object motion clips.

% Subsequently, these clips are fed into the Summarizer, which employs an online motion auto-encoder model known as Stacked Sparse LSTM Auto-Encoder. This model continually updates a customized recurrent auto-encoder network to encode and memorize previous states of object motions. The network's primary task is to reconstruct object-level motion clips, with the reconstruction loss computed between the input and output frames serving as a guide for training the Summarizer.

% Through this training process, the Summarizer becomes proficient in generating summaries that highlight the representative objects in the video and the key motions associated with each object.
\section{Weakly Supervised approaches}
\label{section:rel-weakly}

Weakly-supervised approaches: Similar to unsupervised approaches, weakly-supervised methods aim to reduce the reliance on extensive sets of hand-labeled data. Instead, they utilize less expensive weak labels that are known to be imperfect compared to full human annotations. However, even with these imperfect labels, weakly-supervised approaches can still create robust predictive models for video summarization.

% These categories represent distinct approaches for tackling the challenges associated with training deep learning models for video summarization, and each approach offers its own advantages and considerations. The subsequent sections of this chapter will delve into the specific details of each approach, highlighting key papers and their contributions in advancing the field of video summarization. This comprehensive examination will provide a deeper understanding of the methodologies employed and shed light on the progress made in video summarization research.


% CNNs-based methods have dominated the medical image segmentation task for a long time due to its simplicity and effectiveness. 
% Moreover, Transformers-based methods recently have the attention as a replacement of CNNs-based ones in large-scale datasets; meanwhile, Hybrid methods, which are the combinations of both approaches, are simultaneously under research.
% In this chapter, we go through these above categories in that order with the outstanding methods in each one. 
% Besides, we discuss the limitations of each approach at the end of each section, which leads to the motivation of our proposed method in the next chapter.