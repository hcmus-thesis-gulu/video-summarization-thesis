\section{Objectives}
\label{sec:objectives}

Our main objective is to apply artificial intelligence to the search system to make it \textbf{smarter} (i.e., more flexible queries, more functionalities) and at the same time \textbf{explainable}, while maintaining \textbf{scalability} (i.e., can operate with a large quantity of data).

This is much broader than just building an AI model as we are building a system that consists of many components. For example, using a top-performing yet slow model may forbid the system to run in real-time. A bad user interface might prevent the user from getting any value out of our system at all. Therefore, to achieve our multiple goals, we have to think about the system's architecture and the interaction between various components. Specifically, we look for simple yet effective solutions as they are easier to scale and also simpler to reason about.

With this in mind, our proposed work has the following main contributions:

\begin{itemize}
    \item \textbf{Scalable architecture}: we separate our application into modules to enhance the maintainability and scalability of the whole system. Specifically, we propose the adoption of a \textbf{vector database} to store the embeddings generated by AI models for use in searching, besides using an ordinary database for storing metadata. This approach prevents the AI model from becoming the bottleneck of the system, enabling querying speed comparable to non-AI systems.
    \item \textbf{Practical guidelines}: we compile our extensive experience from varying users to form a list of guidelines, or principles, for users to keep in mind while searching. This gives users some sense of direction while still being flexible enough to adapt to different situations. Our experience is based on a time-pressured setting dealing with difficult queries, so it should be practical enough.
    \item \textbf{Attention on Explainability}: we enhance our system with a novel \textbf{referring expression segmentation} module to precisely point out the object or concept referred to in the image. This strengthens the model's credibility in all scenarios, especially in some critical ones such as medical uses.
\end{itemize}

\vspace{-2mm}
Our system design is further analyzed in Section \ref{chap-first}, the best practices are listed in Section \ref{sec:best_practices}, and lastly our referring expression segmentation sub-system is described in detail in Section \ref{chap-refer-seg}.
 
%  This inspire us to build a \textbf{scalable} system that can handle a large quantity of data while being relatively simple. Moreover, we try to systematize \textbf{best search practices} from our experience as a kind of "user manual" to help guide the user. Finally, we integrate a \textbf{referring expression object segmentation} module to visualize the objects and concepts being mentioned, providing explainability for our system.