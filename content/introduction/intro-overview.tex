\section{Overview}
\label{section:intro-overview}

In recent years, the consumption of video content has experienced a remarkable upsurge, driven by the proliferation of multimedia platforms such as TikTok, YouTube, Instagram, and others. A striking example of this growth can be observed in the case of YouTube, where the number of video content hours uploaded per minute has witnessed a substantial increase. Between 2014 and 2020, there was an approximate 40 percent rise in the rate of uploads, with over 500 hours of video being uploaded every minute as of June 2022 \cite{YoutubeHours}. This surge in video content on platforms like YouTube reflects the expanding demand among consumers for online video consumption. With an approximation of 2.5 quintillion bytes of data created every day \cite{Meena2023Review}, there is a pressing need for effective methods that can automatically generate concise and informative summaries of videos, enabling users to quickly comprehend the content without having to watch the entire video. 

Video summarization, as a research area, focuses on generate concise summaries that effectively capture the temporal and semantic aspects of a video, while preserving its salient content. Achieving this objective involves addressing several fundamental challenges, such as identifying key frames or representative shots, detecting important events, recognizing significant objects or actions, and preserving the overall context and coherence of the video. 

The task plays a crucial role in facilitating efficient browsing, indexing, and retrieval of video data, offering users the ability to preview and comprehend video content without investing significant time and effort. Moreover, video summarization finds applications in various domains, including video surveillance, multimedia retrieval, video archiving, and online video platforms, where it serves as a valuable tool for enhancing user experience and content accessibility \cite{Apostolidis2021Video}.
