\chapter{Thesis Proposal}
\begin{longtable}{|l|c|}
\hline
\multicolumn{2}{|m{\linewidth}|}{\textbf{Thesis title}: MEDICAL IMAGE SEGMENTATION AND DIAGNOSIS
}\\
\hline
\multicolumn{2}{|m{\linewidth}|}{\textbf{Advisors}: Assoc.Prof. Trần Minh Triết, Prof. Đỗ Ngọc Minh} \\
\hline
\multicolumn{2}{|m{\linewidth}|}{\textbf{Duration}: January 2\textsuperscript{nd}, 2019 to June 30\textsuperscript{th}, 2019}\\
\hline
\multicolumn{2}{|m{\linewidth}|}{\textbf{Students}: Hoàng Trung Hiếu (1512159) - Tôn Thất Vĩnh (1512679)}\\
\hline
\multicolumn{2}{|m{\linewidth}|}{\textbf{Theme of Thesis}: theoretical research, proposed improvements.}\\
\hline
\multicolumn{2}{|m{\linewidth}|}{\textbf{Content}:\par
The objective of this thesis is to propose novel methods to improve the performance for two problems in medical image analysis, including abnormalities finding/anatomy landmark detection on endoscopic images and H\&E stained histopathology image segmentation.\par
The details include:
\begin{itemize}
\item Conduct researches on different abnormalities detection on medical image, from handcrafted approaches to deep learning approaches. 

\item Conduct researches on different deep neural networks architecture for image classification and object detection tasks in Computer Vision.

\item Conduct researches on enhancing the performance of deep learning models with data augmentation mechanisms.

\item Conduct experiments of proposed enhancements on endoscopic images and evaluate the results based on main score: Accuracy (ACC) and Matthews Correlation Coefficient (MCC).

\end{itemize}}\\
\hline
\multicolumn{2}{|m{\linewidth}|}{\begin{itemize}

\item Conduct researches on different nuclear segmentation not only on H\&E images but also in other related domains. 

\item Conduct researches on enforcing rotation equivariance in the network, especially in convolutional neural networks.

\item Conduct researches on some semantic segmentation methods, especially instance segmentation.

\item Conduct experiments on proposed enhancements with thoroughly designed ablation study and evaluate the results based on main score: Aggregated Jaccard Index.

\end{itemize}}\\
\multicolumn{2}{|m{\linewidth}|}{
\textbf{Implementation plan}:
\begin{itemize}
\item 02/01/2019-15/01/2019: Literature review about abnormal findings and anatomical landmark detection; nuclear cell segmentation, group-equivariant neural network and related problems.
\item 16/01/2019-01/02/2019: Undertake research on image classification, object detection, instance segmentation methods and challenges in Computer Vision, especially on medical dataset domains.
\item 01/02/2019-15/02/2019: Conduct research about related works on deep-learning based approaches in abnormal findings and anatomical land mark detection; nuclear segmentation.
\item 15/02/2019-28/02/2019: Undertake research on Faster R-CNN, Residual Neural Network, U-Net architectures and related problems.
\end{itemize}}\\
\hline
\multicolumn{2}{|m{\linewidth}|}{
\begin{itemize}
\item 01/03/2019-18/03/2019: Implement an image classification with residual neural network on endoscopic images and enhancement of the long-skip connections in the U-Net with residual blocks.
\item 19/03/2019-15/04/2019: Enhance the training samples with specifically designed data augmentation mechanism for gastrointestinal endoscopic and histological images.
\item 16/04/2019-13/05/2019: Implement an object detection with Faster-R-CNN for abnormal symptoms localization and encode equivariance to groups, specifically rotation and translation, into U-Net architecture.
\item 14/05/2019-15/06/2019: Evaluate the implementations of proposed methods, and make adjustments if necessary.
\item 16/06/2019-30/06/2019: Thesis finalization.
\end{itemize}}\\
\hline
\makecell{\textbf{Advisors} \vspace*{3cm} Trần Minh Triết \hspace{10pt}  Đỗ Ngọc Minh} & \makecell{\textbf{December 26\textsuperscript{th} 2018}\\ \textbf{Authors} \vspace*{2cm} \\Hoàng Trung Hiếu \hspace{10pt} Tôn Thất Vĩnh}\\
\hline
\end{longtable}


