\chapter{Thesis Proposal}
\begin{longtable}{|p{{{80mm}}}|c|}
\hline
\multicolumn{2}{|m{\linewidth}|}{\textbf{Thesis title}: Smart Interaction with Semantic Analysis of Visual Data }\\
\hline
\multicolumn{2}{|m{\linewidth}|}{\textbf{Advisor}: Assoc. Prof. Trần Minh Triết} \\
\hline
\multicolumn{2}{|m{\linewidth}|}{\textbf{Duration}: January 14\textsuperscript{nd}, 2021 to June 30\textsuperscript{th}, 2021}\\
\hline
\multicolumn{2}{|m{\linewidth}|}{\textbf{Student}: Hoàng Xuân Nhật (18125042) - Nguyễn E Rô (18125046)}\\
\hline

\hline
\multicolumn{2}{|m{\linewidth}|}{\textbf{Content}:\par
% Mục tiêu của thesis này là đề xuất giải pháp cho bài toán sinh ra caption mang text tồn tại trong ảnh sao cho hợp lí và giống con người nhất có thể, bên cạnh khả năng ứng dụng thực tế thì nhóm còn nhắm đến cải thiện kết quả trên sota hiện tại.
The target of this thesis is to propose approaches to the Image Captioning with Text problem. Given an image as input, our system first must extract the texts from it, then include them in the generated caption such that it is as reasonable and human-like as possible.
}\\
\hline
\multicolumn{2}{|m{\linewidth}|}{\textbf{Methods}:\par
% Chúng tôi ứng dụng spatial graph cải thiện baseline sẵn có. Sau khi có kết quả, chúng tôi tiếp tục thêm một số bước post process. 
We leverage the available public features and baseline to improve. There are two features needed for our baseline, those are OCR features and Object features. The old baseline forwards given features into Attention layer, Fusion layer, Transformer layer and then use copy mechanism to generate the output caption. Instead of feed only the raw features, we use Graph Convolution Network to embedded the spatial connection between the OCR and Object features and use that as another feature.

}\\
\hline
\multicolumn{2}{|m{\linewidth}|}{\textbf{Results}:\par
% Với task này, baseline gốc đạt, chúng tôi trong giai đoạn thi TextCaps 2021 đạt kết quả ??? và đạt ??? trong challenge . Các thực nghiệm improve thêm dẫn đến kết quả cuối cùng và đạt ???

}\\
\hline

\multicolumn{2}{|m{\linewidth}|}{
\textbf{Research timeline}:
% - January 2021: Thực hiện học và tìm hiểu các kiến thức nền tảng gồm components, metric cho bài toán Image Captioning
% - February 2021: Thực hiện survey các model nổi tiếng, tìm hiểu các workshop liên quan và bộ dataset có thể sử dụng được. % - March - April 2021: Tìm hiểu baseline phù hợp, các phương pháp sử dụng trong baseline, các framework hỗ trợ code được baseline. Tiến hành setup môi trường, kiểm tra framework, reimplement baseline và train thử nghiệm. Bên cạnh đó brainstorm ý tưởng có thể improve baseline.
% - May 2021: tiến hành implement propose methods, train, lấy kết quả, quan sát và xác định các lỗi mang tính pattern. Tham gia TextCaps 2021 và submit kết quả model hiện tại.
% - June 2021: Hoàn thiện conclusion cho các phương pháp improve chưa thực hiện xong. Thử nghiệm các ý tưởng mới nhằm cải thiện những lỗi mang tính pattern đã xác định. Hoàn thiện kết quả code, demo và đưa ra kết quả cuối cùng.

\begin{itemize}
    \item 01/01/2022 to 21/01/2022: Perform literature review on self-supervised
methods
    \item 22/01/2022 to 11/02/2022: Survey vision-language methods
    \item 12/02/2022 to 04/03/2022: Review recent CV and NLP trends
    \item 05/03/2022 to 25/03/2022: Identify data sources
    \item 26/03/2022 to 22/04/2022: Propose the main methodology
    \item 23/04/2022 to 06/05/2022: Data processing and cleaning
    \item 07/05/2022 to 27/05/2022: Survey training and optimization techniques
    \item 28/05/2022 to 24/06/2022: Implementation of our method
    \item 25/06/2022 to 08/07/2022: Evaluation of our models
    \item 09/07/2022 to 31/07/2022: Finalize and build a working demo

\end{itemize}


}\\
\hline

\makecell[c]{\textbf{Advisor} \\ 
\includegraphics[height=2.5cm]{resources/signatures/pmkhoi.png} \\ Assoc. Prof. Trần Minh Triết } & 

\makecell[c]{\textbf{December 15\textsuperscript{th}, 2019}\\ \textbf{Author} \\ 
\includegraphics[height=2.5cm]{resources/signatures/pmkhoi.png} \\ Phạm Minh Khôi \\ 
\includegraphics[height=2.5cm]{resources/signatures/nhtlong.png} \\ Nguyễn Hồ Thăng Long} \\ 
\hline
\end{longtable}


