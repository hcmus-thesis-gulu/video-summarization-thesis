\section{Thesis Content}
\label{section:intro-content}

    After \textbf{Chapter \ref{chapter:introduction}: Introduction}, the remainder of our thesis is composed of 5 chapters as follows:

    \hyperref[chapter:background]{\textbf{Chapter \ref{chapter:background}: Background}}

        In this chapter, we present fundamental knowledge, from Machine Learning and Deep Learning, Neural Networks, Convolutional Neural Networks, to Transformers, which will help us to comprehend the next chapters.

    \hyperref[chapter:related]{\textbf{Chapter \ref{chapter:related}: Related Work}}

        In this chapter, we first provide an overview of three main deep learning approaches for solving video summarization task: supervised method, weakly supervised method, and unsupervised method. At each approach, we discuss the leading paper and explain how the follow-up papers could improve the baseline in many aspects. Finally, we analyze the advantages and disadvantages of each method. 

    \hyperref[chapter:method]{\textbf{Chapter \ref{chapter:method}: Proposed Methods}}

        In this chapter, \dots

    \hyperref[chapter:experiments]{\textbf{Chapter \ref{chapter:experiments}: Experiments}}

        In this chapter, \dots

    \hyperref[chapter:conclusion]{\textbf{Chapter \ref{chapter:conclusion}: Conclusions}}

        In this chapter, \dots


    % \hyperref[chap-preliminary]{\textbf{Chapter \ref{chap-preliminary}: Our Proposed Preliminary Results}}\\
    % In this chapter, we discuss our two previous independent approaches when doing the internship at VinBrain in 3D medical image segmentation in detail. These two approaches were both accepted at the 2022 IEEE 19th
    % International Symposium on Biomedical Imaging (ISBI), which is a great motivation for us in doing this thesis to further improve the results in this task, presented in \hyperref[chap-method]{Chapter \ref{chap-method}: Proposed Method}\\

    % \hyperref[chap-method]{\textbf{Chapter \ref{chap-method}: S3DFormer - 3D Segmentation Method for Medical Analysis with Transformers}}\\
    % In this chapter, we explain our method \textbf{S3DFormer} in details. We talk about the overall motivation and intuition of our method first, then go deeply into our proposed architecture and analyze them under different aspects. 

    % \hyperref[chap-experiment]{\textbf{Chapter \ref{chap-experiment}: Experiment}}\\
    % In this chapter, we perform various experiments to determine the performance of \textbf{S3DFormer} under different benchmarks and metrics. We also compare our proposed method with other SOTA architecture in terms of accuracy and efficiency and make a detailed and precise ablation study to further prove our method in an empirical way.

    % \hyperref[chap-conclusion]{\textbf{Chapter \ref{chap-conclusion}: Conclusion}}\\
    % In this chapter, we summarize our work and briefly discuss the disadvantages of the current approach to pave the way for future research.
    % \label{sec:thesis_content}

    % \textbf{Chapter 2: Related Works}

    % In this chapter, we first introduce an overview of the main approaches for the Interactive Retrieval System and previous versions of our system.
    % We then discuss several methods used in Referring Expression Segmentation tasks. Finally, we discuss state-of-the-art approaches in Semi-supervised Video Object Segmentation that we leverage as a post-process to further boost our performance.

    % \textbf{Chapter 3: Flexible Interactive Retrieval System}

    % In this chapter, we go into detail about our retrieval system called \textbf{FIRST}. First, we outline the overview of our retrieval system. Following sections describe the design of our system and AI-related components in retrieval process. Finally, we provide more details about the Lifelog Search Challenge 2022 and Visual Browser Showdown 2022, the evaluation metrics used in these challenges, and the result of our interactive retrieval system on each challenge.

    % \textbf{Chapter 4: Referring Expression Segmentation}

    % In this chapter, we present our Referring Expression Segmentation framework \textbf{VLFormer} in detail. We first introduce an overview of our framework. Next, we describe each part of our architecture and how efficiently we train the network. Then, an extension to video data is provided. Finally, we discuss about challenges we participated in, the datasets we conducted the experiments, and analyze our performance in both quantitative and qualitative aspects. 

    % \textbf{Chapter 5: Conclusion}

    % In this chapter, we summarize our work and briefly discuss the potential improvements of the current approach for future research.