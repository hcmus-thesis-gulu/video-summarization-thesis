\section{Preliminary}
\label{section:rel-preliminary}

\subsection{Problem Statement}
\label{subsec:rel-statement}

Video summarization aims to generate a concise overview of video content by selecting the most informative and significant parts. The resulting summary can take the form of either a set of representative video frames, known as a video storyboard, or a compilation of video fragments stitched together in chronological order, referred to as a video skim. Video skims have an advantage over static frame sets as they can include audio and motion elements, allowing for a more natural storytelling experience and potentially conveying more information. Moreover, watching a video skim is often more engaging and captivating for viewers compared to a slideshow of frames \cite{Li2001Overview}. On the other hand, storyboards offer greater flexibility in terms of data organization for browsing and navigation purposes, as they are not bound by timing or synchronization constraints \cite{Calic2007Comic,Wang2007VideoCollage}.

Our problem statement aligns closely with the concept of video storyboards, which involve selecting a subset of representative video frames to summarize the content. By focusing on these key frames, we aim to capture the essence and important aspects of the video in a condensed form. This approach allows for efficient browsing and navigation through the video data while providing a comprehensive overview of its content.

\subsection{Problem Formulation}
\label{subsec:rel-formulation}
Given an input video $\textbf{I}=\{I^{(t)}\}_{t=1}^T$ where each frame $I^{(t)} \in \mathbb{R} ^{C \times H \times W}$, the goal of video summarization using the storyboard approach is to generate a concise summary $\textbf{S}$ that consists of a subset of representative frames. The summary $\textbf{S}$ is denoted as $\{I^{(t_i)}\}^k_{i=1}$, where $k$ is typically much smaller than $T$ and $t_1 < t_2 < \dots < t_k$.

\subsection{Datasets}
\label{subsec:rel-datasets}

As referenced in Section \ref{section:intro-motivation}, two datasets that prevail in the video summarization bibliography are SumMe \cite{Gygli2014SumMe} and TVSum \cite{Song2015TVSum}. 

SumMe dataset comprises 25 videos, ranging from 1 to 6 minutes in duration, encompassing diverse content and captured from both first-person and third-person perspectives. Each video has been annotated by 15 to 18 users, resulting in multiple fragment-level user summaries. These summaries typically span 5\% to 15\% of the original video duration. 

TVSum dataset comprises 50 videos, with durations ranging from 1 to 11 minutes. These videos cover content from 10 categories of the TRECVid MED dataset. Each video in TVSum has been annotated by 20 users, providing shot- and frame-level importance scores on a scale of 1 to 5.

In addition to SumMe and TVSum, two common datasets for evaluating video summaries are OVP \cite{De2011VSUMM} and YouTube \cite{De2011VSUMM}. Each dataset comprises 50 videos, with annotations consisting of sets of key-frames generated by 5 users. The video durations span from 1 to 4 minutes for OVP and 1 to 10 minutes for YouTube. These datasets encompass a wide variety of video content, including documentaries, educational videos, ephemeral videos, historical footage, and lectures in the case of OVP, and cartoons, news clips, sports highlights, commercials, TV shows, and home videos in the case of YouTube.

Considering the size of these datasets, it is evident that there is a scarcity of large-scale annotated datasets, which limits their utility in enhancing the training of sophisticated supervised deep learning architectures.

Some less commonly used datasets for video summarization include CoSum \cite{Chu2015CoSum}, MED-summaries \cite{Potapov2014MEDSummaries}, Video Titles in the Wild (VTW) \cite{Zeng2016TitleWild}, League of Legends (LoL) \cite{Fu2017VideoLoL}, and FVPSum \cite{Ho2018FVPSum}. 

CoSum focuses on video co-summarization. It consists of 50 videos obtained from Youtube using 10 query terms related to the content of SumMe dataset. Each video has an approximate duration of 4 minutes, from which sets of key-fragments are selected by 3 different annotators.

MED-Summaries consists of 160 videos from TRECVID 2011 MED dataset. The dataset is divided into a validation set with 60 videos from 15 event categories and a test set with 100 videos from 10 event categories. The majority of videos has durations range from 1 to 5 minutes, with each being annotated with a set of importance scores, averaged over 1 to 4 annotators.

The VTW dataset consists of 18100 open domain videos, out of which 2000 videos are annotated at the sub-shot level with highlight scores. These user-generated videos are untrimmed and typically contain a highlight event. On average, the videos in the dataset have a duration of 1.5 minutes.

The LoL dataset consists of 218 long videos, ranging from 30 to 50 minutes in duration. These videos showcase game matches from the North American League of Legends Championship Series (NALCS). The annotations for this dataset are derived from a YouTube channel that features community-generated highlights, with the highlight videos typically having a duration of 5 to 7 minutes. As a result, there is one set of key-fragments available for each video in the dataset.

The FPVSum dataset focuses on first-person video summarization and comprises 98 videos, totaling over 7 hours of content. These videos are sourced from 14 categories of GoPro viewer-friendly videos. For each category, approximately 35\% of the video sequences have been annotated with ground-truth scores by at least 10 users, while the remaining sequences are considered unlabeled examples. This dataset provides valuable resources for evaluating and developing first-person video summarization algorithms.

Apostolidis \etal~\cite{Apostolidis2021Video} have compiled a comprehensive summarization table, showcasing the main characteristics of the aforementioned datasets. For reference, \hyperref[table:dataset-characteristics]{Table \ref{table:dataset-characteristics}} presents an overview of the dataset attributes, such as video count, annotation types, video duration, and user involvement.
\begin{table}
  \caption{Datasets for video summarization and their characteristics.}
  \scriptsize
  \begin{tabular}{|M{0.09\textwidth}|M{0.07\textwidth}|M{0.09\textwidth}|M{0.4\textwidth}|M{0.16\textwidth}|M{0.11\textwidth}|}
    \hline
    \bfseries Dataset & \bfseries no. videos & \bfseries duration (min) & \bfseries content & \bfseries type of annotations & \bfseries annotators per video \\ 
    [0.5ex] 
    \hline\hline
    SumMe \cite{Gygli2014SumMe} & 25 & 1 - 6 & holidays, events, sports & multiple sets of key-fragments & 15 - 18 \\
    \hline
    TVSum \cite{Song2015TVSum} & 50 & 2 - 10 & news, how-to's, user-generated, documentaries (10 categories - 5 video each) & multiple fragment level scores & 20 \\
    \hline
    OVP \cite{De2011VSUMM} & 50 & 1 - 4 & documentary, educational, ephemeral, historical, lecture & multiple sets of key frames & 5 \\
    \hline
    YouTube \cite{De2011VSUMM} & 50 & 1 - 10 & cartoons, sports, tv-shows, commercial. home videos & multiple sets of key frames & 5 \\
    \hline
    CoSum \cite{Chu2015CoSum} & 51 & ~ 4 & holidays, events, sports (10 categories) & multiple sets of key fragments & 3 \\
    \hline
    MED \cite{Potapov2014MEDSummaries} & 160 & 1 - 5 & 15 categories of various genres & one set of importance score & 1 - 4 \\
    \hline
    VTW \cite{Zeng2016TitleWild} & 2000 & 1.5 (avg) & user-generated videos that contain a highlight event & sub-shot level highlight scores & - \\
    \hline
    LoL \cite{Fu2017VideoLoL} & 218 & 30 - 50 & matches from a League of Legends tournament & one set of key fragments & 1 \\
    \hline
    FPVSum \cite{Ho2018FVPSum} & 98 & 4.3 (avg) & first-person videos (14 categories) & multiple frame level scores & 10 \\
    \hline
  \end{tabular}
  \label{table:dataset-characteristics}
\end{table}

% In this study, we will thoroughly analyze and employ these datasets as benchmarks for evaluating the performance of video summarization algorithms.

\subsection{Evaluation Metrics}
\label{subsec:rel-evaluation}