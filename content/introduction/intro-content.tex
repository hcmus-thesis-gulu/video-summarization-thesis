\section{Thesis Content}
\label{section:intro-content}

    After \textbf{Chapter \ref{chapter:introduction}: Introduction}, the remainder of our thesis is composed of 5 chapters as follows:

    \hyperref[chapter:background]{\textbf{Chapter \ref{chapter:background}: Background}}

        In this chapter, we provide a comprehensive overview of fundamental concepts in Machine Learning and Deep Learning. We also introduce some basic models that will be used in the subsequent chapters. This foundational knowledge is essential for understanding the content covered in the following chapters.

    \hyperref[chapter:related]{\textbf{Chapter \ref{chapter:related}: Related Work}}

        In this chapter, we first provide an overview of three main deep learning approaches for solving video summarization task: supervised method, weakly supervised method, and unsupervised method. At each approach, we discuss the leading paper and explain how the follow-up papers could improve the baseline in many aspects. Finally, we analyze the advantages and disadvantages of each method. 

    \hyperref[chapter:method]{\textbf{Chapter \ref{chapter:method}: Proposed Methods}}

        In this chapter, we provide a detailed explanation of each of our methods, starting with the baseline version and then discussing the improved versions. We begin by outlining the overall motivation and intuition behind our approach. Subsequently, we delve into the specifics of our proposed architecture and analyze its components from various perspectives.

    \hyperref[chapter:experiments]{\textbf{Chapter \ref{chapter:experiments}: Experiments}}

        In this chapter, we conduct comprehensive experiments to evaluate our proposed methods using both qualitative and quantitative approaches. We compare our approach with state-of-the-art architectures in terms of accuracy and efficiency, highlighting its strengths. Furthermore, we provide a detailed and precise ablation study to further validate the effectiveness of our method empirically.

    \hyperref[chapter:conclusion]{\textbf{Chapter \ref{chapter:conclusion}: Conclusions}}

        In this chapter, we summarize our work and briefly discuss the disadvantages of the current approach to pave the way for future research.